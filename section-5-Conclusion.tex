\section{Summary and future plans}

PhEDEx has been very successful at managing data-flows on the WAN for the CMS 
collaboration. Nonetheless, its architecture is based on design decisions that 
start to become invalid, and in order to continue to scale and perform for the 
future, it must evolve, taking advantage of new technologies.

Over the course of the past two years, the ANSE project has made significant 
progress towards integrating network awareness into PhEDEx. A prototype was created and 
tested , demonstrating the essential features required for circuit integration 
into PhEDEx. Using the lessons learned during the prototype development, a new improved 
version was designed, which is going to be released in production shortly. 
The new system is capable of functioning independently from PhEDEx, it can be 
controlled remotely via a REST interface and it is modular, being able to use 
multiple circuit providers.

The remaining work in ANSE, focuses more on the network aspect of the solution.
While the PhEDEx software is ready to make use of circuits as soon as they mature 
into a production ready version, the interim solution still relies on more work 
on our part, mainly in extending a circuit from storage to storage. 
This is where ANSE will invest its resources in the following months.