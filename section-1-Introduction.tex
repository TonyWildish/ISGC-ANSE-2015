\section{Introduction}
\subsection{PhEDEx}

PhEDEx\cite{PhEDEx} is the data-placement management tool for the CMS\cite{CMS} 
experiment at the LHC. It manages the scheduling of all large-scale WAN transfers 
in CMS, ensuring reliable delivery of the data. It consists of several components:

\begin{itemize}
	\item an Oracle database, hosted at CERN
	\item a website and data-service, which users (humans or machine) use to 
	interact with and control PhEDEx
	\item a set of \emph{central} agents that deal with routing, request-management,
	bookeeping and other activities. These agents are also hosted at CERN, though 
	they could be run anywhere. The key point is that there is only one set of 
	central agents per PhEDEx instance
	\item a set of \emph{site-agents}, one set for every site that receives data
\end{itemize}

PhEDEx was originally conceived over ten years ago now, and the architecture still 
reflects design decisions made at that time. Then, the network was expected to be 
the weakest link in the developing Worldwide LHC Grid (WLCG)\cite{WLCG}. Networks 
were expected to have bandwidth of the order of 100 Mb/sec, to be unreliable, and 
to be poorly connected across the span of the CMS experiment. Accordingly, PhEDEx 
will back off fast and retry gently in the face of failed transfers, on the assumption 
that failures will take time to fix, and that there is other data that can be 
transferred in the meantime. This can lead to large latencies caused by transient 
errors, with subsequent delays in processing the data.

The data-transfer topology was designed with a strongly hierarchical structure. 
The Tier-0 (CERN) transferred data primarily to a set of 6-7 Tier-1 sites, and each
 Tier-1 site handled traffic between itself, the other Tier-1s. and it's local Tier-2
 sites. Tier-2's wishing to exchange data would have to go via Tier-1 intermediaries. 
 This kept the transfer-links (i.e. the set of (source,destination) pairs) mostly in 
 the realm of a single regional network operator, the only cross-region links were 
 used by Tier-1s which were assumed to have the expertise to debug problems and keep 
 the data flowing. It also kept the overall number of transfer links low, since the 
 majority of sites (the Tier-2s) had only one link, to their associated Tier-1 site.
 
Today, the reality is very different. The network has emerged as the most reliable 
component of the WLCG; problems with transfers tend to be at the end-points rather 
than in the network itself. Bandwidths of 10 Gb/sec between Tier-2 sites is common 
in many areas, and 100 Gb/sec connectivity is starting to appear. Even where the 
bandwidth is still relatively low, connections are quite reliable, so data can be 
transferred effectively. This has led CMS to embrace a fully connected transfer mesh 
in which all sites are allowed to connect to any other site, so the number of transfer 
links has risen from about 100 to nearer 3000.

\subsection{ANSE}

CMS is now considering a number of possible avenues for the future of PhEDEx\cite{TW_DB_CHEP13},
with ANSE\cite{ANSE}, exploring the use of dynamic circuits. ANSE is an NSF CC-NIE funded project, 
which started in January 2013 for an initial period of two years. It is a collaboration between 
4 universities (Caltech, Vanderbilt University, University of Michigan and University of Texas at 
Arlington) which in the context of CMS seek to make PhEDEx more aware of the network status, 
and allow it to control the network by way of virtual circuits and bandwidth-on-demand (BoD).